\documentclass[a4paper]{article}

\usepackage[english]{babel}
\usepackage[utf8]{inputenc}

\title{Relatório - Mini EP 6 - Aplicando \emph{design patterns}}

\author{
Guilherme Schützer - NUSP 8658544 \\
Ricardo Lira       - NUSP 8536131
}

\date{18/11/2014}

\begin{document}
\maketitle

\section{Preparação}


	Inicialmente, nós tivemos que decidir em algum programa/código e um Design Pattern
para aplicar. Após várias decisões e escolhas, chegamos em um projeto que fizemos
ano passado em MAC122: Um IMDB (Internet Movie DataBase) simplificado, que recebe
arquivos de texto como entrada com vários filmes e informações adicionais como``nota''
e ``ano que foi lançado''. Junto com essa escolha decidimos utilizar do Design Pattern \emph{Strategy}, focando em uma das características bacanas do programa: Os diferentes meios de ordenação da lista que foram/podem ser implementados.

	O programa original possuia duas opções de \emph{sort}, que eram pra ser implementadas por nós: \emph{MergeSort} e \emph{QuickSort}. Com o padrão \emph{Strategy} na cabeça nós decidimos deixar de um jeito mais eficiente a utilização de novos \emph{sorts} no programa, e especialmente caso seja preciso adicionar novos.

	O primeiro desafio foi a linguagem. Nosso exercício programa foi feito inicialmente em C,
que por não ser uma linguagem orientada a objetos, dificulta um pouco aplicar de forma concreta os conhecimentos e funcionalidades do padrão \emph{Strategy} que vimos em sala de aula. Decidimos então primeiro traduzir o código para uma linguagem que melhor podemos usufruir das vantagens de \emph{design pattern}, e que condiz com nosso curso atual: Python.

\newpage

\section{Transcrição e aplicação do \emph{design pattern}}

	Para a transcrição do código, optamos por manter apenas as funcionalidades necessárias para a aplicação do padrão \emph{Strategy}. Isso significa a interação com a lista de filmes, ou seja, criação/remoção da elementos da lista, por \emph{input} manual ou por leitura de um arquivo (claro que, em Python, isso requer muito menos linhas - e menos ainda manipulação de
	ponteiros e liberação de memória), e a opção de ordenação.

	Assim, o contexto criado para o padrão compreende duas classes, \texttt{ListaFilmes} e \texttt{Filme}, para lidar com objetos de filme e lista. As implementações das classes, por sua vez, são praticamente iguais às respectivas implementações em C no código original. Contudo, com a nova versão do código, é possível trabalhar com mais de uma instância de lista, o que permite comparações e, consequentemente, maior riqueza de testes.

	Para a aplicação do padrão, foi utilizando o módulo \texttt{ABC} da biblioteca padrão do Python, que permite a criação de classes abstratas explicitamente. Há no código a implementação de um ordenador padrão \texttt{StandardSort} que usa o método \texttt{sort()} da biblioteca padrão do Python.



\end{document}

